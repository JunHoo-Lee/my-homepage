\documentclass{article}

% if you need to pass options to natbib, use, e.g.:
%     \PassOptionsToPackage{numbers, compress}{natbib}
% before loading neurips_2025

% The authors should use one of these tracks.
% Before accepting by the NeurIPS conference, select one of the options below.
% 0. "default" for submission
 \usepackage[main, final]{neurips_2025}
% the "default" option is equal to the "main" option, which is used for the Main Track with double-blind reviewing.
% 1. "main" option is used for the Main Track
%  \usepackage[main]{neurips_2025}
% 2. "position" option is used for the Position Paper Track
%  \usepackage[position]{neurips_2025}
% 3. "dandb" option is used for the Datasets & Benchmarks Track
 % \usepackage[dandb]{neurips_2025}
% 4. "creativeai" option is used for the Creative AI Track
%  \usepackage[creativeai]{neurips_2025}
% 5. "sglblindworkshop" option is used for the Workshop with single-blind reviewing
 % \usepackage[sglblindworkshop]{neurips_2025}
% 6. "dblblindworkshop" option is used for the Workshop with double-blind reviewing
%  \usepackage[dblblindworkshop]{neurips_2025}

% After being accepted, the authors should add "final" behind the track to compile a camera-ready version.
% 1. Main Track
 % \usepackage[main, final]{neurips_2025}
% 2. Position Paper Track
%  \usepackage[position, final]{neurips_2025}
% 3. Datasets & Benchmarks Track
 % \usepackage[dandb, final]{neurips_2025}
% 4. Creative AI Track
%  \usepackage[creativeai, final]{neurips_2025}
% 5. Workshop with single-blind reviewing
%  \usepackage[sglblindworkshop, final]{neurips_2025}
% 6. Workshop with double-blind reviewing
%  \usepackage[dblblindworkshop, final]{neurips_2025}
% Note. For the workshop paper template, both \title{} and \workshoptitle{} are required, with the former indicating the paper title shown in the title and the latter indicating the workshop title displayed in the footnote.
% For workshops (5., 6.), the authors should add the name of the workshop, "\workshoptitle" command is used to set the workshop title.
% \workshoptitle{WORKSHOP TITLE}

% "preprint" option is used for arXiv or other preprint submissions
 % \usepackage[preprint]{neurips_2025}

% to avoid loading the natbib package, add option nonatbib:
%    \usepackage[nonatbib]{neurips_2025}

\usepackage[utf8]{inputenc} % allow utf-8 input
\usepackage[T1]{fontenc}    % use 8-bit T1 fonts
\usepackage{hyperref}       % hyperlinks
\usepackage{url}            % simple URL typesetting
\usepackage{booktabs}       % professional-quality tables
\usepackage{amsfonts}       % blackboard math symbols
\usepackage{nicefrac}       % compact symbols for 1/2, etc.
\usepackage{microtype}      % microtypography
\usepackage{xcolor}         % colors
\usepackage{kotex}
\usepackage{graphicx}

% Note. For the workshop paper template, both \title{} and \workshoptitle{} are required, with the former indicating the paper title shown in the title and the latter indicating the workshop title displayed in the footnote. 
\title{Discriminative Ranking Optimization is all you need in Multiple-Choice Question Answering}


% The \author macro works with any number of authors. There are two commands
% used to separate the names and addresses of multiple authors: \And and \AND.
%
% Using \And between authors leaves it to LaTeX to determine where to break the
% lines. Using \AND forces a line break at that point. So, if LaTeX puts 3 of 4
% authors names on the first line, and the last on the second line, try using
% \AND instead of \And before the third author name.


\author{%
  Junhoo Lee\thanks{https://junhoo.me} \\
  Seoul National University \\
  \texttt{mrjunoo@snu.ac.kr} \\
  % examples of more authors
  % \And
  % Coauthor \\
  % Affiliation \\
  % Address \\
  % \texttt{email} \\
  % \AND
  % Coauthor \\
  % Affiliation \\
  % Address \\
  % \texttt{email} \\
  % \And
  % Coauthor \\
  % Affiliation \\
  % Address \\
  % \texttt{email} \\
  % \And
  % Coauthor \\
  % Affiliation \\
  % Address \\
  % \texttt{email} \\
}


\begin{document}


\maketitle


\begin{abstract}
Standard Supervised Fine-Tuning (SFT) for Multiple-Choice Question (MCQ) reasoning tasks suffers from a fundamental objective mismatch. While SFT optimizes the probability distribution over the global vocabulary to enforce generative fluency, MCQ evaluation relies strictly on the local ranking of candidate options. Consequently, SFT inefficiently allocates parameter capacity to correcting surface-level behaviors rather than sharpening the decision boundary between correct answers and distractors. To address this, we propose {Discriminative Ranking Optimization (DRO)}, a parameter-efficient alignment strategy that transforms the training paradigm from generation to discrimination. DRO restricts the optimization search space by aligning training templates with the evaluation protocol and utilizing intrinsic distractors to maximize the relative margin of the ground truth. Empirical results on the ARC-Challenge demonstrate that our approach elevates the zero-shot baseline of 61.43\% to \textbf{74.40\%}, significantly outperforming standard SFT. Furthermore, we observe a phenomenon of {mechanism transfer}, where a model trained on an unrelated commonsense dataset achieves comparable performance on scientific reasoning. This suggests that DRO enables the model to internalize a generalized discriminative mechanism independent of domain-specific knowledge. Code and data are available at \url{junhoo.me/DRO}.
\end{abstract}


\section{Introduction}

% Paragraph 1: Task Definition - Emphasizing "Closed-Set Ranking"
The domain of Question Answering (QA), specifically the AI2 Reasoning Challenge (ARC), requires reasoning capabilities that extend beyond simple fact retrieval. Unlike open-ended generation tasks, the evaluation protocol for ARC operates as a {closed-set ranking task}. The model is not required to generate a solution from scratch but must strictly assign the highest likelihood to the correct option among a constrained set of candidates. Consequently, the primary objective is to accurately {discriminate} the ground truth from distractors rather than prioritizing global generative fluency.

% Paragraph 2: Challenges - From "SFT Limits" to "Objective Mismatch"
Standard Supervised Fine-Tuning (SFT) encounters a structural {objective mismatch} in this context. SFT optimizes the probability distribution over the entire global vocabulary. This process allocates significant parameter capacity to global correction, such as suppressing formatting artifacts, rather than sharpening the decision boundary between candidates. This inefficiency explains why injecting reasoning traces, as seen in SlimOrca, or incorporating heavy domain knowledge, like Mathstral, often yields diminishing returns. Lacking explicit negative feedback, SFT models prioritize syntactic adherence over the {discriminative logic} necessary to distinguish truth from high-quality distractors.

% Paragraph 3: Proposed Method - Introducing DRO
We address these challenges under Parameter-Efficient Fine-Tuning (PEFT) constraints by proposing {Discriminative Ranking Optimization (DRO)}. This methodology shifts the training paradigm from generation to discrimination by minimizing the optimization search space. We first implement an {Evaluation-Aware Template Strategy} that aligns training inputs with the evaluation protocol to freeze the syntactic search space. Additionally, we utilize intrinsic distractors from the dataset as hard negatives. The model is trained to strictly maximize the {relative margin} between the correct answer and the distractors, ensuring that limited parameters are dedicated exclusively to the ranking objective.

% Paragraph 4: Results & Mechanism Transfer
We empirically validate the efficacy of DRO through extensive experiments. Our method achieved a substantial performance improvement on the ARC-Challenge, increasing accuracy from a zero-shot baseline of 61.43\% to {74.40\%}. Beyond task-specific gains, we present evidence of {mechanism transfer}. A model trained exclusively on the unrelated {CommonsenseQA} dataset achieved performance on ARC comparable to a model trained on the target data itself. This result suggests that DRO equips the model with a {fundamental discriminative mechanism}, or a generalized skill for multiple-choice reasoning, rather than merely overfitting to domain-specific knowledge.
% % Paragraph 1: Context & Task Definition
% The domain of Scientific Question Answering (QA), specifically the ARC (AI2 Reasoning Challenge) benchmark, requires models to perform complex reasoning beyond simple fact retrieval. However, contrary to open-ended generation tasks, the evaluation protocol for ARC is fundamentally a {ranking task}: the model must assign the highest likelihood to the correct option among multiple choices. Consequently, the core objective of training is not merely to generate coherent text, but to accurately {discriminate} the correct answer from incorrect ones based on likelihood scores.

% % Paragraph 2: The Challenges (SFT Limits, LIMA, Format Bias)
% However, standard Supervised Fine-Tuning (SFT) faces structural limitations in fostering this discriminative capability. Recent studies, such as the LIMA hypothesis, suggest that SFT primarily functions as ``surface form alignment'' rather than injecting new knowledge or significantly enhancing reasoning capabilities. This limitation is exacerbated in Multiple-Choice Question (MCQ) tasks, where the evaluation relies on single-token likelihoods without the intermediate supervision of reasoning steps (Chain-of-Thought). Lacking explicit negative feedback, SFT models often struggle to distinguish correct logic from {plausible distractors} and exhibit brittle performance due to {template sensitivity} and {positional bias}. Thus, simply maximizing the probability of the ground truth is insufficient to sharpen the decision boundary against high-quality distractors.

% % Paragraph 3: Proposed Method (Strategic Alignment & Evaluation-Aware Template)
% To address these challenges under the constraints of limited compute (PEFT), we propose a {``Strategic Alignment''} methodology that shifts the training objective from generation to discrimination. First, we employ a {Preference Optimization (specifically DPO/SimPO)} framework tailored for MCQs. By utilizing the dataset's intrinsic distractors as ``Hard Negative'' signals, we train the model to explicitly maximize the margin between scientific truth and plausible falsehoods. Crucially, we implement an {Evaluation-Aware Template Strategy}. Unlike standard SFT which imposes an arbitrary format, we align our training templates exactly with the downstream evaluation protocol. This design minimizes the ``alignment tax,'' ensuring that the limited parameter budget of PEFT is allocated exclusively to sharpening the {discriminative boundary} rather than adapting to surface-level syntax.

% % Paragraph 4: Results & Generalization (ARC + CommonsenseQA)
% We empirically validate the efficacy of our approach through extensive experiments. Our method achieved a substantial performance leap on the ARC-Challenge, elevating the accuracy from a baseline of 61.43\% to {74.xx\%}. Beyond task-specific improvements, we present compelling evidence of the model's {generalization capabilities}. Despite being trained \textit{exclusively} on the ARC-Challenge training set, our model demonstrated remarkable transfer learning effects on other multiple-choice benchmarks. Notably, on {CommonsenseQA}, accuracy surged from 59.3\% to {73.5\%} without any direct exposure to the dataset. This result strongly suggests that our approach equips the model with a {fundamental discriminative mechanism} robust enough to handle unseen multiple-choice tasks, rather than merely overfitting to domain-specific knowledge.
\section{Related Work}

\paragraph{Instruction Tuning for Reasoning}
Enhancing the reasoning capabilities of Large Language Models (LLMs) has primarily relied on Supervised Fine-Tuning (SFT) with high-quality demonstration data. Methods like SlimOrca or Mathstral utilize synthetic reasoning traces (e.g., Chain-of-Thought) generated by stronger models to guide smaller models. While these approaches improve general instruction following and generative coherence, they inherently treat the task as open-ended generation. As highlighted in our analysis, this generative objective often misaligns with the strict ranking nature of multiple-choice evaluations, leading to suboptimal parameter efficiency and potential negative transfer when domain-specific data is over-utilized.

\paragraph{Preference Learning in Reasoning Tasks}
Recent advancements have shifted towards Preference Optimization methods, such as RLHF and DPO, to align models with human intent. In the context of reasoning, several studies have applied DPO to improve performance on benchmarks like MMLU. However, the prevailing paradigm in these works is to optimize the quality of \textit{intermediate reasoning steps} or generative explanations. Improvements in multiple-choice accuracy are typically reported as auxiliary benefits or {by-products} of enhanced reasoning capabilities. These methods still operate under a generative framework, aiming to produce better text rather than explicitly sharpening the decision boundary between pre-defined options.

\paragraph{Direct Optimization for Multiple-Choice QA}
Our work diverges from the aforementioned approaches by re-framing Scientific QA not as a reasoning generation problem, but as a {discriminative ranking problem}. Unlike methods that rely on generated negatives or generic preference pairs, we leverage the intrinsic distractors provided within the dataset to construct hard-negative pairs. This allows us to apply reference-free preference optimization (SimPO) to directly target the relative margin between the correct answer and plausible falsehoods. \textbf{To the best of our knowledge, this is the first work to explicitly decouple the discriminative ranking objective from generative reasoning in the context of scientific QA, demonstrating that mechanism transfer can be achieved without domain-specific knowledge injection.}
% \section{Method}

% \subsection{Pipeline?}

% 여기서 할 말은 뭐냐면, 우리가 뭘 사용할거고, 그래서 왜 이것만 가능한지. 그러니까, RAG 쓸수도 있을거고, 아니면 고전적으로 여러개 뽑아서 voting 할수도 있을거고, 아니면 reasoning할 수도 있을거고.. 여러가지 방법 가능할것같은데 실제로는 왜 그냥 peft만 했는지. 왜냐하면 실제로는 표준화된 pipeline에서 검증을 할 거고, (왜냐하면 이 말을 왜 하는거냐면, 지금 업스테이지 인턴중인건데 여기서 하는게 foundational model에서 성능뽑는게 중요함) 그럼 그런 트릭같은걸 쓰는게 의미가 없다.

% \subsection{Challenges}

% 여기서 할 말은 뭐냐면, Scientific QA, 그냥 multiple QA 세팅에서의 문제점들. 그리고 실제로 SFT를 통해서 얻을 것. 그러니까 SFT signal을 준다고 해보자. 이떄 문제가 뭐냐면.. 그니까 그 답 자체가 만약에 format일수도 있으니까. 그 말은 뭐냐면 그냥 회문이어서. 아예 이상한 답을 할 수도 있고... 이러니까. 이 경우엔 어떤 템플릿 이해못하기? 영역일수도 있어서 애매하다.

% 일단 내가 본게 뭐냐면, 처음에는 reasoning 에서는 그러면 보통 reasoning 능력 늘려주면 좋다고 하니까 찾아본게 뭐냐면, slimorca. 그게 뭐냐면, reasoning tradce를 gpt4 에서 만들어놓고, 그것들 가지고 fine-tune 한 모델. 그 경우에 arc score가  62.54. 그렇다면, 이제 문제는 어떻게 바뀌는거냐면, 사실 reasoning 자체가 성능을 엄청나게 올려주지는 않는다. 적어도. 그 내 가용 자원 상에서는 그렇다. 왜냐하면, slimorca가 제대로 된 distillation data, 그리고 Full fine tuning. 그렇다면 이거 이상으로 reasoning을 늘리기는 어렵다.
% 추가적인 observation이. 그렇다면 과연 이론상으로 엄청나게 수학쪽으로 학습된 mathtral은 어떨까? 그거마저도 58.64. 오히려 이 경우는 성능이 더 떨어졌다. 여기서 잠정적으로 내린 결론은, 나이브하게 저런 세팅에서는 하면 안된다는 거.

% 이는 왜 그런거냐면, 우리의 가설은, 실제로 과학을 더 학습시키는 건 불가능하고. 이때 해야하는거는 그냥 주어진 train data의 구조 자체를 잘 활용하는 거가 최선. 즉 그냥 객관식 문제에 대해서 더 잘 답하는 문제로 변환하기. 즉. 그냥 주어진 Q,A가 있으면. 여러개의 choice 중에 더 강한 choice만 주도록 그냥 weight 주기. 즉 내가 하는거는, 가용가능한 weight pool에서, 애초에 search space를 줄여서 유의미한 학습공간이 가능하도록 하는 것.즉, 포맷팅 이런 류의 공간은 보지 않겠다.

\begin{figure}
    \centering
    \includegraphics[width=1\linewidth]{image.png}
    \caption{Illustration of the Objective Mismatch. The standard next-token prediction objective (left) requires the model to suppress irrelevant global vocabulary tokens, whereas the ARC protocol (right) only necessitates relative likelihood comparisons between predefined candidates. We hypothesize that this discrepancy causes inefficient allocation of model capacity, hindering performance on closed-set tasks.}
    \label{fig:placeholder}
\end{figure}
\begin{table}[ht]
\centering
\caption{Comparison of model performance on the ARC-Challenge benchmark. Despite reasoning-heavy or domain-specific fine-tuning, improvements remain marginal or negative compared to the base model.}
\label{tab:arc_performance}
\begin{tabular}{llc}
\toprule
\textbf{Model} & \textbf{Key Characteristics} & \textbf{ARC-C (Acc, \%)} \\ \midrule
\textbf{Mistral-7B} & General-purpose base model & \textbf{61.43} \\
SlimOrca & Fine-tuned on high-quality reasoning traces (SFT) & 62.64 \\
Mathstral & Specialized in STEM domains; Heavy knowledge injection & 58.64 \\ \bottomrule
\end{tabular}
\end{table}
\section{Methodology}

\subsection{Experimental Scope}
Our primary objective aligns with the fundamental goal of foundation model development\textbf{(독파모)}, which is to enhance the {intrinsic reasoning capabilities} of the model without reliance on external augmentations. Assuming that evaluations are conducted through standardized protocols, we strictly exclude inference-time techniques such as Retrieval-Augmented Generation (RAG) or Majority Voting ensembles. These methods are typically incompatible with standard evaluation pipelines and can obscure the standalone performance of the model. Therefore, we focus on optimizing the model weights under strict resource constraints. We utilize {PEFT (LoRA)} to demonstrate that performance gains are derived from {efficient alignment strategies} rather than massive compute scaling.

\subsection{Challenges: The Objective Mismatch in SFT}
Before designing our approach, we analyzed the structural limitations of standard Supervised Fine-Tuning (SFT) in the context of Multiple-Choice Question (MCQ) tasks. Our empirical observations from preliminary experiments suggest a fundamental {misalignment between the training objective and the evaluation metric}.

\paragraph{Limitations of Reasoning Injection}
One might hypothesize that the performance bottleneck stems from a lack of reasoning capability. If this were true, supervised fine-tuning on high-quality reasoning traces should yield significant improvements. However, \textit{SlimOrca}, which was fine-tuned on reasoning data with GPT-4, showed similar returns compared to the baseline. This suggests that merely exposing the model to reasoning steps via SFT is insufficient to enhance performance on discriminative benchmarks.

\paragraph{Inefficacy of Domain-Specific SFT}
Alternatively, if the limitation were a lack of domain knowledge, models specialized in scientific domains should outperform generalist models. However, \textit{Mathstral}, an architecture heavily fine-tuned for STEM tasks, underperformed compared to our base model, \textit{Mistral}. This indicates that naive knowledge injection can lead to negative transfer, where domain specialization compromises general commonsense reasoning.

\paragraph{Global Correction vs. Local Ranking}
The standard SFT objective (Next-Token Prediction) optimizes the probability distribution over the entire global vocabulary ($|V| \approx 32k+$). In this regime, a significant portion of the parameter capacity is allocated to {behavioral correction}. This involves suppressing non-compliant tokens, such as junk words or formatting artifacts like `[EOS]` and `\textbackslash n`, to enforce syntactic adherence.

Consider a concrete example from the dataset: \textit{"George wants to warm his hands quickly by rubbing them. Which skin surface will produce the most heat?"} with candidates \{ \textit{Palms, Head, Heart} \}.
Under the SFT objective, the model creates gradients to maximize the probability of the token \textit{"Palms"} while simultaneously suppressing thousands of irrelevant tokens such as \texttt{[}, \texttt{Template:}, or \texttt{Answer:}. 
However, the ARC evaluation protocol is strictly a {closed-set ranking task}. The metric is agnostic to whether $P(\textit{Palms}) > P(\texttt{[EOS]})$; it solely requires that $P(\textit{Palms}) > P(\textit{Head})$ and $P(\textit{Palms}) > P(\textit{Heart})$.

In this context, optimizing the model to "speak correctly" by suppressing global vocabulary noise represents a computational inefficiency. 


\subsection{Discriminative Ranking Optimization}

To bypass the computational inefficiency of global generative correction, we propose a strategy of \textbf{Discriminative Ranking Optimization (DRO)}. This approach strictly optimizes the {relative margin} within the candidate set, effectively disregarding irrelevant global generative artifacts. We implement this through two key mechanisms:

\begin{itemize}
    \item \textbf{Syntactic Constraint (Template Alignment):} We align the training templates exactly with the downstream evaluation protocol. This effectively freezes the syntactic search space and removes the necessity for the model to learn arbitrary instruction formats. This strategy mitigates the "alignment tax," preventing the model from forgetting knowledge while adapting to new syntax.
    
    \item \textbf{Local-Ranking Based Optimization:} Instead of optimizing for global probability across the entire vocabulary, we utilize the dataset's intrinsic distractors to explicitly construct negative pairs. This redefines the loss landscape: the objective is strictly to maximize the {relative likelihood margin} between the correct option and the distractors. This ensures the model learns to rank the valid answer higher than plausible falsehoods within the closed candidate set, without wasting capacity on generative fluency.
\end{itemize}

By decoupling syntax from semantics, we ensure that every gradient update contributes solely to {sharpening the discriminative boundary} within the candidate set. This maximizes the efficiency of the limited trainable parameters available in the PEFT setting.

\paragraph{Instantiation}
We instantiated the proposed strategy with specific design choices to maximize efficiency under the PEFT constraint. 
First, to enforce the {Syntactic Constraint}, we pre-processed the training dataset to strictly mirror the prompt templates used in the \texttt{lm-evaluation-harness}. This ensures zero distribution shift between the training and evaluation phases. 
Second, for the {Local-Ranking Based Optimization}, we adopted {Simple Preference Optimization (SimPO)}, a reference-free variant of Contrastive Preference Optimization (CPO). We selected SimPO over standard DPO because it directly enforces a target reward margin without the computational overhead of a reference model, aligning perfectly with our resource-constrained setting. 
Finally, the preference pairs $(y_{chosen}, y_{rejected})$ were explicitly constructed by assigning the ground truth option as the chosen response and the provided {distractors} as the rejected responses, thereby training the model to discriminate against the specific pitfalls designed by the dataset authors.
% \subsection{Format Aware Policy Optimization}
% \section{Methodology}

% \subsection{Experimental Scope: Focusing on Intrinsic Capabilities}
% Our primary objective aligns with the \textbf{fundamental goal of foundation model development}(독파모): enhancing the model's {intrinsic reasoning capabilities} without reliance on external augmentations. assuming the evaluattions are measured through 규격화된 protocol. Consequently, we exclude inference-time techniques such as Retrieval-Augmented Generation (RAG) or Majority Voting ensembles. While these methods can artificially inflate scores, these methods are cannot used in these protocol. Therefoe, we focus on optimizing the model weights under resource constraint, utilizing \textbf{PEFT (LoRA)} to demonstrate that performance gains are derived from {efficient alignment strategies} rather than massive compute scaling.

% \subsection{Challenges: The Objective Mismatch in SFT}
% Before designing our approach, we analyzed the structural limitations of standard Supervised Fine-Tuning (SFT) in the context of Multiple-Choice Question (MCQ) tasks. Our empirical observations from preliminary experiments—where \textit{SlimOrca} (Reasoning SFT) and \textit{Mathstral} (Domain SFT) yielded diminishing returns or negative transfer—suggest a fundamental \textbf{misalignment between the training objective and the evaluation metric}.

% \paragraph{Reasoning Does not Help} 만약 우리가 이 문제가 추론능력ㅇ르 향상시켜서 해결될 수 있을거라 생각한다면, reasoning 능력을 늘려서 해결될 수 있을거라 생각할 수 이ㅓㅅ다.

% \paragraph{Massive SFT does not help} mathstral은 scientific discovery 문제를 풀기위해서 STEM 잘풀기 위해서 파인튠 많이 된 아키텍쳐다. 만약에 추가 지식을 집어넣는게 효과적이라면, mathstral, 즉 우리의 base model인 mistral에서 finetune된게 성능이 좋아야 한다.

% \paragraph{MCQ is local ranking problem} 
% The standard SFT objective (Next-Token Prediction) optimizes the probability distribution over the entire global vocabulary ($|V| \approx 32k+$). In this regime, a significant portion of the parameter capacity is allocated to \textbf{behavioral correction}—suppressing non-compliant tokens (e.g., junk words, formatting artifacts like `[EOS]` or `\n`) to enforce syntactic adherence.

% Consider a concrete example from the dataset: \textit{"George wants to warm his hands quickly by rubbing them. Which skin surface will produce the most heat?"} with candidates \{ \textit{Palms, Head, Heart} \}.
% Under the SFT objective, the model creates gradients to maximize the probability of the token \textit{"Palms"} while simultaneously suppressing thousands of irrelevant tokens such as \texttt{[}, \texttt{Template:}, or \texttt{Answer:}. 
% However, the ARC evaluation protocol is strictly a \textbf{closed-set ranking task}. The metric is agnostic to whether $P(\textit{Palms}) > P(\texttt{[EOS]})$; it solely requires that $P(\textit{Palms}) > P(\textit{Head})$ and $P(\textit{Palms}) > P(\textit{Heart})$. 

% In this context, optimizing the model to "speak correctly" by suppressing global vocabulary noise represents a computational inefficiency. Our strategy bypasses this by utilizing DPO to strictly optimize the \textbf{relative margin} within the candidate set, ignoring irrelevant global generative artifacts.


% \begin{itemize}
%     \item \textbf{Syntactic Constraint (Template Alignment):} We align the training templates exactly with the downstream evaluation protocol. This effectively "freezes" the syntactic search space, removing the need for the model to learn arbitrary instruction formats. This prevents the alignment tax where the model forgets knowledge while adapting to new syntax.
    
%     \item \textbf{Discriminative Ranking Optimization:} Instead of SFT, we employ Direct Preference Optimization (DPO) utilizing the dataset's intrinsic distractors as negative pairs. This redefines the loss landscape: rather than maximizing the absolute probability of the target token against the entire vocabulary, the objective is strictly to maximize the \textbf{relative margin} between the correct option and the distractors.
% \end{itemize}

% By decoupling syntax from semantics, we ensure that every gradient update contributes solely to \textbf{sharpening the discriminative boundary} within the candidate set, thereby maximizing the efficiency of the limited trainable parameters.
\begin{figure}
    \centering
    \includegraphics[width=1\linewidth]{algorithm.png}
    \caption{Our Algorithm. Discriminative Ranking Optimization (DRO)}
    \label{fig:placeholder}
\end{figure}
\section{Results}

\subsection{Experimental Setup}
We utilized the standard {ARC-Challenge} and {ARC-Easy} training splits provided by the AI2 dataset. The evaluation was conducted using the \texttt{lm-evaluation-harness} framework, reporting \texttt{acc\_norm} (normalized accuracy) to ensure consistency with public leaderboards.
We compare three distinct settings:
\begin{itemize}
    \item \textbf{Base Model:} The pre-trained Mistral-7B model evaluated in a zero-shot setting.
    \item \textbf{Baseline (SFT):} The base model fine-tuned using standard SFT with the "Syntactic Constraint" (template alignment) applied.
    \item \textbf{Ours (DRO):} The proposed method trained on the ARC dataset.
\end{itemize}

\subsection{Main Results: ARC-Challenge}
Table \ref{tab:main_results} presents the comparative performance on the ARC-Challenge benchmark. We use the Base Model's zero-shot performance as the reference point.

\begin{table}[h]
    \centering
    \caption{{Main Results on ARC-Challenge.} Our Local-Ranking Optimization achieves a significant performance boost. The \textcolor{green}{{Green}} values denote the absolute improvement ($\Delta$) over the Base Model.}
    \vspace{0.2cm}
    \begin{tabular}{l|c|c|c}
        \toprule
        \textbf{Method} & \textbf{Training Data} & \textbf{Acc (Norm)} & \textbf{$\Delta$ Gain} \\
        \midrule
        Base Model (Zero-shot) & None & 61.43\% & - \\
        Baseline (SFT + Template Align) & ARC-Challenge & 64.85\% & \textcolor{green}{+3.42} \\
        \textbf{Ours (DRO)} & \textbf{ARC-Challenge} & \textbf{74.40\%} & \textcolor{green}{\textbf{+12.97}} \\
        \bottomrule
    \end{tabular}
    \label{tab:main_results}
\end{table}

The standard SFT baseline reached \textbf{64.85\%}, showing a marginal improvement of +3.42\%p over the base model. In contrast, our Local-Ranking Optimization strategy achieved \textbf{74.40\%}, a substantial improvement of \textbf{+12.97\%p}. This empirical evidence strongly supports our hypothesis: given the same data and model capacity, optimizing for the \textit{discriminative margin} is significantly more sample-efficient than optimizing for \textit{generative likelihood}.

\subsection{Analysis of Generalization and Mechanism Transfer}
Beyond the main task performance, we conducted cross-domain evaluations to verify whether the model learned domain-specific "knowledge" or a task-agnostic "ranking mechanism."

\paragraph{Zero-Shot Generalization (ARC $\rightarrow$ CQA)}
We evaluated the model trained on ARC-Challenge against the \textbf{CommonsenseQA (CQA)} benchmark to measure zero-shot generalization. As shown in Table \ref{tab:cqa_results}, our method demonstrates remarkable transfer capabilities.

\begin{table}[h]
    \centering
    \caption{\textbf{Zero-Shot Generalization on CommonsenseQA.} Despite being trained exclusively on ARC (Scientific QA), the model exhibits a significant performance leap on CommonsenseQA (General QA), indicating robust transfer of the discriminative mechanism.}
    \vspace{0.2cm}
    \begin{tabular}{l|c|c}
        \toprule
        \textbf{Method} & \textbf{Training Data} & \textbf{CommonsenseQA Score} \\
        \midrule
        Base Model (Zero-shot) & None & 59.30\% \\
        \textbf{Ours (DRO)} & \textbf{ARC-Challenge (Only)} & \textbf{73.46\%} \\
        \bottomrule
    \end{tabular}
    \label{tab:cqa_results}
\end{table}

The performance improved from the base level of 59.30\% to \textbf{73.46\%}. This result confirms that the discriminative capability learned from scientific questions transfers seamlessly to general commonsense reasoning, suggesting that the model has internalized a generalized ranking skill rather than overfitting to scientific facts.

\paragraph{Mechanism Transfer (CQA $\rightarrow$ ARC)}
To rigorously test this "ranking skill" hypothesis, we trained a model exclusively on {CommonsenseQA (CQA)}—a dataset logically distinct from ARC—and evaluated it on ARC-Challenge.

\begin{table}[h]
    \centering
    \caption{{Mechanism Transfer Analysis.} Notably, our method trained on unrelated data (CQA) matches the performance of SFT trained on the target data (ARC).}
    \vspace{0.2cm}
    \begin{tabular}{l|c|c}
        \toprule
        \textbf{Method} & \textbf{Training Data} & \textbf{ARC-Challenge Score} \\
        \midrule
        Baseline (SFT) & ARC (Target Domain) & 64.85\% \\
        \textbf{Ours (Cross-Domain)} & \textbf{CommonsenseQA (Unrelated)} & \textbf{64.25\%} \\
        \bottomrule
    \end{tabular}
    \label{tab:mechanism_transfer}
\end{table}

As shown in Table \ref{tab:mechanism_transfer}, the model trained on CQA achieved \textbf{64.25\%} on ARC-Challenge. Despite being trained on {unrelated data} (general commonsense vs. scientific facts), the model matched the performance of the SFT baseline trained on the target domain (64.85\%).
This confirms that the {Local-Ranking Optimization} effectively sharpens the model's intrinsic ability to distinguish between plausible options, regardless of the specific subject matter. It suggests that the performance gain stems not merely from memorizing domain facts, but from mastering the {structural mechanism of multiple-choice discrimination}.


\section{Conclusion}
In this work, we identified the structural inefficiency of standard Supervised Fine-Tuning (SFT) for multiple-choice reasoning tasks and proposed a {Local-Ranking Optimization} strategy to address the misalignment between the generative training objective and the discriminative evaluation protocol. By constraining the syntactic search space via template alignment and directly optimizing the relative margin between candidates, we achieved a substantial performance improvement on the ARC-Challenge, elevating accuracy from 61.43\% to {74.40\%} under strict PEFT constraints. Beyond task-specific gains, our cross-domain experiments demonstrated that this approach enables the model to acquire a generalized {discriminative mechanism} rather than merely memorizing domain knowledge, as evidenced by the robust zero-shot transfer to CommonsenseQA. These findings underscore that unlocking the intrinsic reasoning capabilities of foundation models depends less on massive data injection and more on the precise alignment of the optimization objective with the downstream task structure.

\end{document}